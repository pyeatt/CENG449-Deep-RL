\documentclass[12pt]{article}

\textwidth 6.5in
\textheight 9in
\evensidemargin 0in
\oddsidemargin 0in
\topmargin 0in
\headheight 0in
\headsep 0in

\usepackage{times}
\usepackage{hyperref}
\usepackage{tabularx}
\newcolumntype{Y}{>{\centering\arraybackslash}X}

\usepackage{graphicx}
\usepackage{color}

\usepackage{listings}
\newlength{\listwidth}
\setlength{\listwidth}{\textwidth}
\addtolength{\listwidth}{-20pt}

\newlength{\smallbasewidth}
\settowidth{\smallbasewidth}{\ttfamily\small m}
\newlength{\footnotebasewidth}
\settowidth{\footnotebasewidth}{\ttfamily\footnotesize m}
\newlength{\scriptbasewidth}
\settowidth{\scriptbasewidth}{\ttfamily\scriptsize m}

\makeatletter
\lstdefinestyle{mystyle}{
  basewidth=\footnotebasewidth,
  basicstyle=\ttfamily\lst@ifdisplaystyle\ttfamily\footnotesize\fi
}
\makeatother


\lstset{language=C,style=mystyle,frame=tlrb,framesep=10pt,linewidth=\listwidth,xleftmargin=12pt,xrightmargin=-8pt,framerule=2pt}

\begin{document}
\begin{center}
  {
  \bf\large
  CSC  449/549 --- Advanced Topics in Artificial Intelligence\\
  Deep Reinforcement Learning
  }\\
  Fall, 2022\\
\ \\
    {\bf
      Alternative Final Exam
    }
  
\end{center}
Implement Sarsa($\lambda$) for the Pole Balancing problem as described
in Sutton and Barto. Use linear function approximation. You can use any method you want to create the feature vector, including but not limited to:
\begin{itemize}
\item Fourier
basis
functions,
\item radial basis functions,
\item polynomial basis functions,
\item deep learning techniques.
\end{itemize}

Write a report to describe your approach, citing any sources as appropriate. 
Your report should follow a typical conference research paper outline with the following sections:
\begin{itemize}
  \item Abstract
  \item Introduction/Background
  \item Related Work
  \item Description of experiments
  \item Results
  \item Comments and Future Work
  \item References
\end{itemize}
You should show at least nine learning curves (multiple curves can be shown on the same graph) for various parameter settings. Your report should be between 6 and 10 pages in length, ten to twelve point font, single-spaced, one or two columns.


%\item Create a surface plot of the value function (the negative of the value function) of the learned policies after 1, 000 episodes,
%for the above orders. (Hint: Your plot should look like the one in Sutton and Barto, but smoother.)


  %The Mountain Car contains a negative step reward and a zero goal reward. What would happen if $\gamma$ was less than
%1 and the solution was many steps long? What would happen if we had a zero step cost and a positive goal reward,
%for the case where $\gamma = 1$, and the case where $\gamma < 1$?
%\end{enumerate}

Turn in your report and your code through D2L.


\end{document}
